\documentclass[12pt]{article}

% Русский язык
\usepackage[english, russian]{babel}

% Красная строка
\usepackage{indentfirst}\frenchspacing

% Размер страницы и поля
\usepackage[a4paper, portrait, top=2cm, left=3cm, bottom=2cm, textwidth=16cm]{geometry}

% Шрифты
\usepackage{fontspec}
\setmainfont{Times New Roman}

% Интерлиньяж
\usepackage{setspace} 
\setstretch{1.2}

% Метаданные PDF и настройки гиперссылок
\usepackage[
  pdfauthor={Монки Луффи Драгонович},
  pdftitle={Отчет учебной (эксплуатационной) практики}
]{hyperref}
\hypersetup{colorlinks=true, linkcolor=black}

% Таблицы с автоподбором ширины столбцов
\usepackage{array}
\usepackage{tabularx}

\begin{document}

\thispagestyle{empty}
\begin{center}
  Министерство науки и высшего образования Российской Федерации \\
  ФГАОУ ВО <<Северо-Востоный федеральный университет имени М.К. Аммосова>> \\
  Институт математики и информатики \\
  Кафедра <<Информационные технологии>>
\end{center}

  \vfill
  
\begin{center}
  \Large{
    \textbf{ОТЧЕТ} \\
    \textbf{учебной (эксплуатационной) практики}
  }
\end{center}

\vfill

\begin{flushleft}
Выполнил: \underline{Монки Луффи Драгонович} \\
Направление подготовки: \underline{09.03.01 <<Информатика и вычислительная техника>>,}\\ \underline{профиль <<Технологии разработки программного обеспечения>>} \\
Группа: \underline{Б-ИВТ-23} \\
Вид практики: \underline{Учебная (эксплуатационная)} \\
Сроки прохождения практики: \underline{14 июня -- 27 июня 2024 г.} \\
Место прохождения практики: \underline{кафедра <<Информационные технологии>>} \\
Руководитель практики: \underline{Никифоров Дьулустан Васильевич, \; ст. преп. каф.  <<Информационные технологии>>}
\end{flushleft}

\vfill

\begin{center}
  Якутск, 2024
\end{center}


\newpage \section*{Введение}

Здесь укажите цели и задачи практики. Цели --- это то, чего хочется достичь: развить навыки, разобраться в рабочих процессах. Задачи --- это конкретные действия, которые вы должны выполнить по ходу учебной практики.

Также во введении можно добавить, почему данная практика важна для вашей учебы и будущей карьеры.


\section{Онлайн-курс <<LaTeX с нуля>>}

Краткая характеристика онлайн-курса и описание приобретенных знаний, умений навыков.


\section{Онлайн-курс <<Markdown>>}

Краткая характеристика онлайн-курса и описание приобретенных знаний, умений навыков.


\section{Онлайн-курс <<Введение в Git>>}

Краткая характеристика онлайн-курса и описание приобретенных знаний, умений навыков.


\section{Самостоятельная работа}



\section{Обработка текста для датасета}

Цель работы и объем выполненной работы.


\section*{Заключение}

В заключении, кратко опишите выполненные задачи и достигнутые цели. Объясните, что вам понравилось или не понравилось в изученных технологиях и как вы планируете их использовать в будущем.


\newpage \section*{Дневник учебно-эксплуатационной практики}

\noindent
\begin{tabularx}{\textwidth}
{| >{\centering\arraybackslash}p{0.5cm} | >{\centering\arraybackslash}X | 
   >{\centering\arraybackslash}p{3.5cm} | >{\centering\arraybackslash}p{2.5cm} |}
\hline
\textbf{№} & \textbf{Содержание работ} & \textbf{Вид отчета} & \textbf{Даты выполнения работ} \\ \hline
1  & Прохождение курса <<LaTeX с нуля>> & Сертификат & 14.06 -- 16.06 \\ \hline
2  & Прохождение курса <<Markdown>> & Сертификат & 17.06 \\ \hline
3  & Прохождение курса <<Введение в Git>> & Скриншот & 18.06 -- 19.06 \\ \hline
4  & Самостоятельная работа & Github репозиторий & 20.06 -- 23.06  \\ \hline
5  & Обработка текста для датасета & Текстовые файлы & 24.06 -- 26.06 \\ \hline
6  & Оформление отчета & tex + pdf & 27.06.2024 \\ \hline
\end{tabularx}

\vspace{1cm}

\noindent
\begin{tabularx}{\textwidth}{ X >{\centering\arraybackslash}X >{\raggedleft\arraybackslash}X }
Подпись практиканта: & & / Монки Л.Д. / \\
& $\overline{\parbox[t]{3cm}{\centering\footnotesize(подпись)}}$ \vspace{1cm} & \\
\multicolumn{3}{l}{Содержание и объём выполненных работ подтверждаю:} \\
Руководитель практики: & & / Никифоров Д.В. / \\
& $\overline{\parbox[t]{3cm}{\centering\footnotesize(подпись)}}$ \vspace{1cm} & \\
Оценка практики: & \underline{\hspace{3cm}} & \\
& $\overline{\parbox[t]{3cm}{}}$ & \\
\end{tabularx}

    
\newpage \section*{Приложения}

\end{document}